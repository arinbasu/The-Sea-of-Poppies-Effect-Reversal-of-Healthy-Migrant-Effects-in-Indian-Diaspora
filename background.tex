## Survey 

Towards the end of the 20th century, when the western world was preparing itself for the possible risk of computation and database management because of the way dates were handled at the time (termed as the YZK database threat), to overcome and address this issue of computing, it required input from programmers across the world in a cost efficient manner. At that time, it opened the doors for India as an important back-office destination of the world, where hundreds of expert computer programmers and other information technology professionals were employed by the global corporations to address this threat. Since then, India's information technology enabled services (ITES) and business process outsourcing (BPO) industries  have never looked back, and it continues to grow. According to a NASSCOM-McKinsey study  conducted in 2007, the Indian ITES/BPO industry grew 7 times the annual GDP growth. In 2009, the industry contributed nearly 2.5 percent to the GDP (need to add a citation).

    
    
    