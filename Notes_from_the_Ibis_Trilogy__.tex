\subsection{Notes from the Ibis Trilogy (Sea of Poppies)}


\subsection{The purpose of Ibis the Ship}

"One thing Zachary did know about the Ibis was that she had been built to serve as a ‘blackbirder’, for transporting slaves. This, indeed, was the reason why she had changed hands: in the years since the formal abolition of the slave trade, British and American naval vessels had taken to patrolling the West African coast in growing numbers, and the Ibis was not swift enough to be confident of outrunning them. As with many another slave-ship, the schooner’s new owner had acquired her with an eye to fitting her for a different trade: the export of opium. In this instance the purchasers were a firm called Burnham Bros., a ship-ping company and trading house that had extensive interests in India and China"
Note the nature of the ship. The point to note is how or why did the ship sail here and what drreams were sold? What were the decptions?

\subsection{Lascars or the sea farers }


(pp. 13)

"He had thought that lascars were a tribe or nation, like the Cherokee or Sioux: he discovered now that they came from places that were far apart, and had nothing in common, except the Indian Ocean; among them were Chinese and East Africans, Arabs and Malays, Bengalis and Goans, Tamils and Arakanese. They came in groups of ten or fifteen, each with a leaderwhospoke on their be-half. Tobreak up these groups was impossible; they had to be taken together or not at all, and although they came cheap, they had their own ideas of how much work they would do and how many men would share each job - which seemed to mean that three or four las-cars had to be hired for jobs that could well be done by a single able seaman"

\subsection{Zachary's Ignorance (pp. 14)}

\begin{quote}
"once, when the serang spat a stream of blood-red juice over the rail, he noticed the water below coming alive with the thrashing of shark’s fins. How harmless could this betel-stuff be if it could be mistaken for blood by a shark?"
\end{quote}
\begin{quote}
Mark of Zachary's ignorance
\end{quote}

\subsection{Hukam Singh's Illness (pp. 22)}


"Two days later, Deeti and her daughter were eating their midday meal when Chandan Singh stopped his ox-cart at their door. Kabutri-ki-ma! he shouted. Listen: Hukam Singh has passed out, at the factory. They said you should go there and bring him home. With that he gave his reins a snap and drove off hurriedly, impatient for his meal and his afternoon sleep: it was typical of him to offer no help. A chill crept up Deeti’s neck as she absorbed this: it was not that the news itself was totally unexpected - her husband had been ailing for some time and his collapse did not come entirely as a surprise. Rather, her foreboding sprang from a certainty that this turn of events was somehow connected with the ship she had seen; it was as if the very wind that was bearing it towards her had blown a draught up her spine. Ma? said Kabutri. What shall we do? How will we bring him home?"
This is a significant turn of event and note the sense of abandonment. This guy just shrugs his shoulder and passes the bad news and moves on. Then there is this tension with the charioteer who is low caste. we now get to see that circumstances are driving this family to be accosted to the ship

Neel had but recently come into the title, ... (pp. 30)
"Neel had but recently come into the title, having inherited it upon his father’s death two years before: he was in his late twenties, and although well past his first youth, he retained the frail, etiolated frame"
Neel Rattan Halder was a young man

Mr Doughty snorted contemptuously. ‘These days it takes ... (pp. 35)
"Mr Doughty snorted contemptuously. ‘These days it takes no more than an acre or two for a Baboo to style himself a More-Roger. And the way this one jaws on, you’d think he’s the Padshaw of Persia. Wait till you hear the barnshoot bucking in English - like a bandar reading aloud from The Times.’ He chuckled gleefully, twirling the knob of his cane. ‘Now that’ll be something else to look forward to this evening, apart from the chitchky - a spot of bandarbaiting.’ He paused to give Zachary a broad wink. ‘From what I hear, the Rascal’s going to be in for a samjaoing soon enough. The kubber is that his cuzzanah is running out.’"
the proximity of the Raja of Rashkali to the English and their lifestyle, he also knew English

A boat will take them to Patna and ... (pp. 51)
"A boat will take them to Patna and then to Calcutta, said the guard. And from there they’ll go to a place called Mareech. Unable to restrain herself any longer, Deeti joined in the conversation, asking, from the shelter of her sari’s ghungta: Where is this Mareech? Is it near Dilli? Ramsaran-ji laughed. No, he said scornfully. It’s an island in the sea - like Lanka, but farther away. The mention of Lanka, with its evocation of Ravana and his demon-legions, made Deeti flinch. How was it possible that the marchers could stay on their feet, knowing what lay ahead? She tried to imagine what it would be like to be in their place, to know that you were forever an outcaste; to know that you would never again enter your father’s house; that you would never throw your arms around your mother; never eat a meal with your sisters and brothers; never feel the cleansing touch of the Ganga. And to know also that for the rest of your days you would eke out a living on some wild, demon-plagued island? Deeti shivered. And how will they get to that place? she asked Ramsaran-ji. A ship will be waiting for them at Calcutta, said the duffadar, a jahaz, much larger than any you’ve ever seen: with many masts and sails; a ship large enough to hold hundreds of people …"
Selling of the dream

Calcutta was then the principal conduit through which ... (pp. 54)
"Calcutta was then the principal conduit through which Indian prisoners were shipped to the British Empire’s network of island prisons - Penang, Bencoolen, Port Blair and Mauritius. Like a great stream of silt, thousands of Pindaris,Thugs, dacoits, rebels, head-hunters and hoo-ligans were carried away by the muddy waters of the Hooghly to be dispersed around the Indian Ocean, in the various island jails where the British incarcerated their enemies"
Fate of the prisoners -- people were transported from Calcutta to the different islands in the Indian Ocean

Closing the ship’s log, Mr Burnham turned his ... (pp. 55)
"Closing the ship’s log, Mr Burnham turned his attention to the correspondence that had accumulated over the course of the voyage. M. d’Epinay’s letter from Mauritius seemed particularly to catch his interest, especially after Zachary reported the planter’s parting words about his sugar-cane rotting in the fields and his desperate need for coolies"
Need for Coolies in he ship

The Ibis won’t be carrying opium on her ... (pp. 55)
"The Ibis won’t be carrying opium on her first voyage, Reid. The Chinese have been making trouble on that score and until such time as they can be made to understand the benefits of Free Trade, I’m not going to send any more shipments to Canton. Till then, this vessel is going to do just the kind of work she was in-The suggestion startled Zachary: ‘D’you mean to use her as a slaver, sir? But have not your English laws outlawed that trade?’ ‘That is true,’ Mr Burnham nodded. ‘Yes indeed they have, Reid. It’s sad but true that there are many who’ll stop at nothing to halt the march of human freedom.’ ‘Freedom, sir?’ said Zachary, wondering if he had misheard"
Slavery and the otehr side's view

A hold that was designed to carry slaves ... (pp. 56)
"A hold that was designed to carry slaves will serve just as well to carry coolies and convicts. Do you not think? We’ll put in a couple of heads and piss-dales, so the darkies needn’t always be fouling themselves. That should keep the in-spectors happy."
transporting coolies

His doubts were quickly put at rest. ‘Freedom, ... (pp. 56)
"His doubts were quickly put at rest. ‘Freedom, yes, exactly,’ said Mr Burnham. ‘Isn’t that what the mastery of the white man means for the lesser races? As I see it, Reid, the Africa trade was the greatest exercise in freedom since God led the children of Israel out of Egypt. Consider, Reid, the situation of a so-called slave in the Carolinas - is he not more free than his brethren in Africa, groaning under the rule of some dark tyrant?’ Zachary tugged his ear-lobe. ‘Well sir, if slavery is freedom then I’m glad I don’t have to make a meal of it. Whips and chains are not much to my taste.’ ‘Oh come now, Reid!’ said Mr Burnham. ‘The march to the shining city is never without pain, is it? Didn’t the Israelites suffer in the desert?"
Slavery and the otehr side's view.

Not slaves - coolies. Have you not heard ... (pp. 56)
"Not slaves - coolies. Have you not heard it said that when God closes one door he opens another? When the doors of freedom were closed to the African, the Lord opened them to a tribe that was yet more needful of it - the Asi-atick.’"
Not slaves but coolies

in eastern India, opium was the exclusive monopoly ... (pp. 59)
"in eastern India, opium was the exclusive monopoly of the British, produced and packaged en-tirely under the supervision of the East India Company; except for a small group of Parsis, few native-born Indians had access to the trade or its profits"
Opium Trading in the hand of East India Company

Some twenty-five years before, when his trading house ... (pp. 59)
"Some twenty-five years before, when his trading house was still in its infancy, Mr Benjamin Burnham had come to see the old Raja with an eye to leasing one of his properties as an office: he needed a Dufter but was short on capital, he said, and would have to defer the payment of the rent. Unbeknownst to Mr Burnham, while he was presenting his case, a white mouse had appeared under his chair - hidden from the trader, but perfectly visible to the zemindar, it sat still until the Englishman had had his say. A mouse being the famili-ar of Ganesh-thakur, god of opportunities and remover of obstacles, the old zemindar had taken the visitation to be an indication of divine will: not only had he allowed Mr Burnham to defer his rent for a year, he had also imposed the condition that the Raskhali estate be allowed to invest in the fledgling agency"
How Neel Rattan halder's father invested in Benjamin Burnham's Estate -- the illogical bits of the investment

the Halders had built their fortunes over the ... (pp. 59)
"the Halders had built their fortunes over the last cen-tury and a half. In the era of the Mughals, they had ingratiated themselves with the dyn-asty’s representatives; at the time of the East India Company’s arrival, they had extended a wary welcome to the newcomers; when the British went to war against the Muslim rulers of Bengal, they had lent money to one side and sepoys to the other, waiting to see which would prevail. After the British proved victorious, they had proved as adept at the learn-ing of English as they had previously been in the acquisition of Persian and Urdu. When it was to their advantage, they were glad to shape their lives to the world of the English; yet they were vigilant always to prevent too deep an intersection between the two circles."

It was only in the final days of ... (pp. 60)
"It was only in the final days of his life that the old zemindar informed his son that the family’s financial survival depen-ded on their dealings with Mr Benjamin Burnham; the more they invested with him the better, for their silver would come back doubled in value. He explained that in order to make the best of this arrangement, he had told Mr Burnham that this year he would like to venture the equivalent of one lakh sicca rupees. Knowing that it would take time to raise such a large amount, Mr Burnham had kindly offered to forward a part of the sum from his own funds: the understanding was that the money would be made good by the Halders if it wasn’t covered by the profits from that summer’s opium sales."
Beginning of the snare 1

A few days later the old zemindar had ... (pp. 61)
"A few days later the old zemindar had died, and with his passing everything seemed to change. That year, 1837, was the first in which Burnham Bros. failed to generate profits for its clients. In the past, when the opium ships returned from China, at the end of the trad-ing season, Mr Burnham had always come in person to the Raskhali Rajbari - the Halders’ principal seat in Calcutta. It was the custom for the Englishman to bring auspicious gifts, like areca-nuts and saffron, as well as bills and bullion. But in the first year of Neel’s in-cumbency there was neither a visit nor any money: instead the new Raja received a letter informing him that the China trade had been severely affected by the sudden decline in the value of American bills of exchange; its losses aside, Burnham Bros. was now facing severe difficulties in remitting funds from England to India. At the end of the letter there was a polite note requesting the Raskhali estate to make good on its debts."
The trapping of Rasakali RajbaRi, and how Benjamin Burnham tightened the noose aorund the neck of the Raja Neel Rattan 2

the old Raja had always got on well ... (pp. 61)
"the old Raja had always got on well with Englishmen, even though he spoke their language imperfectly and had no interest at all in their books. As if to compensate for his own limitations, the Raja had hired a British tutor for his son, to make sure that he had a thorough schooling in English. This tutor, Mr Beasley, had much in common with Neel, and had encouraged his interests in literature and philosophy. But far from putting him at ease in the society of Calcutta’s Englishmen, Neel’s education had served exactly the opposite end. For Mr Beasley’s sensibility was unusual amongst the Brit-ish colonials of the city, who tended to regard refinements of taste with suspicion, and even derision - and never more so than when they were evinced by native gentlemen."
The colonials contempt of the natives and the natives urge to learn the language

When Mr Burnham’s note arrived, the Halder mansion ... (pp. 61)
"When Mr Burnham’s note arrived, the Halder mansion was already under siege by an army of less-er creditors: some of these were wealthy merchants, who could, without compunction, be staved off for a while; but there were also many relatives and underlings who had entrusted what little they had to the zemindar - these impoverished and trusting dependants could not be refused. It was in trying to return their money that Neel discovered that his estate had no more cash available than was required to cover its expenses for a week or two. The situation was such that he demeaned himself to the point of sending a pleading letter to Mr Burnham, asking not just for time, but also a loan to tide the estate through until the next season"
The trapping of Raskali RajbaRi, part 3, the role of lesser donors and how the Burnhams trapped the Rajahs.

Sudder Opium Factory - fondly spoken of by ... (pp. 63)
"Sudder Opium Factory - fondly spoken of by old Company hands as the ‘Ghazee-pore Carcanna’. The factory was immense: its premises covered forty-five acres and sprawled over two adjoining compounds, each with numerous courtyards, water tanks and iron-roofed sheds. Like the great medieval forts that overlooked the Ganga, the factory was so situated as to have easy access to the river while being high enough to escape sea-sonal floods. But unlike such forts as Chunar and Buxar, which were overgrown and largely abandoned, the Carcanna was anything but a picturesque ruin: its turrets housed squads of sentries, and its parapets were manned by a great number of peons and armed burkundazes. The day-to-day management of the factory was in the hands of a superintendent, a senior official of the East India Company who oversaw a staff of several hundred Indian workers: the rest of the British contingent consisted of overseers, accountants, storekeep-ers, chemists and two grades of assistant. The superintendent lived on the premises, and his sprawling bungalow was surrounded by a colourful garden, planted with many variet-ies of ornamental poppy."
The Opium Factory. Remember the opium trade was in the hands of the British; the Indians would have none of it. They were just workers and petty people. The trade and the superintendence was in the hands of the Briitish officers.

Leaving Kabutri in Kalua’s cart, Deeti headed alone ... (pp. 64)
"Leaving Kabutri in Kalua’s cart, Deeti headed alone towards the factory’s entrance, nearby. Here stood the weighing shed to which the farmers of the district brought their poppy-leaf wrappers every spring, to be weighed and sorted into grades of fine and coarse, ‘chandee’ and ‘ganta’. This was where Deeti would have sent her own rotis, had she accu-mulated enough to make the trouble worthwhile. Around harvest time there was always a great press of people here, but the crop being late this year, the crowd was relatively small. A small troop of uniformed burkundazes was on duty at the gate, and Deeti was relieved to see that their sirdar, a stately whitemoustachioed elder, was a distant relative of her hus-band’s. When she went up to him and murmured Hukam Singh’s name, he knew exactly why she had come. Your husband’s condition isn’t good, he said, ushering her into the fact-ory. Get him home quickly"
Deeti arrives at the opium factory

once again Deeti was taken aback by the ... (pp. 65)
"once again Deeti was taken aback by the space ahead, but this time not because of the vastness of its dimensions, but rather the opposite - it was like a dim tunnel, lit only by a few small holes in the wall. The air inside was hot and fetid, like that of a closed kit-chen, except that the smell was not of spices and oil, but of liquid opium, mixed with the dull stench of sweat - a reek so powerful that she had to pinch her nose to keep herself from gagging. No sooner had she steadied herself, than her eyes were met by a startling sight -a host of dark, legless torsos was circling around and around, like some enslaved tribe of demons. This vision - along with the overpowering fumes - made her groggy, and to keep herself from fainting she began to move slowly ahead. When her eyes had grown more ac-customed to the gloom, she discovered the secret of those circling torsos: they were bare-bodied men, sunk waist-deep in tanks of opium, tramping round and round to soften the sludge."
Men, who were working in the opium processing,manual labourers to feed their masters greed. How different it is this view from the exploitation of the call centres of today where men and women toil, often losing their own names (the names have to sound like Christian names so that they are familiar to foreign ears; their work schedules and hours need to match the comfortable hours or demands of their western masters in call centres; note the similarity between these working conditions and the conditions of the modern day call centres? forget the glitz; it's all the same. But hte fundamental question is, why?

So this was where they came, these offspring ... (pp. 65)
"So this was where they came, these offspring of her fields? Deeti could not help looking around in curiosity, marvelling at the speed and dexterity with which the vessels were whisked on and off the scales. Then, with paper battas attached, they were carried to a seated sahib, who proceeded to poke, prod and sniff their contents before marking them with a seal, allowing some through for processing, and condemning others to some lesser use. Nearby, held back by a line of lathi-carrying peons, stood the farmers whose vessels were being weighed; alternatively tense and angry, cringing and resigned, they were waiting to find out if their harvests for the year had ful-filled their contracts - if not, they would have to start the next year with a still greater load of debt. Deeti watched as a peon carried a slip of paper to a farmer and was rebuffed with a howl of protest: all over the hall, she noticed, there were quarrels and altercations breaking out, with farmers shouting at serishtas, and landlords berating their tenants"
Colonialism, how the Englishmen treated the men in the land where they came to do business; the submission of the native before the market

Deeti’s escort whispered in her ear: This is ... (pp. 66)
"Deeti’s escort whispered in her ear: This is where the afeem is brought in to dry, after it’s been assembled. She noticed now that the shelves were joined by struts and ladders; glancing around, she saw troops of boys clinging to the timber scaffolding, climbing as nimbly as acrobats at a fair, hopping from shelf to shelf to examine the balls of opium. Every now and again, an English overseer would call out an order and the boys would begin to toss spheres of opium to each other, relaying them from hand to hand until they had come to rest safely on the floor. How could they throw so accurately with one hand, while holding on with the other - and that too at a height where the slightest slip would mean certain death? The sureness of their grip seemed amazing to Deeti, until suddenly one of them did indeed drop a ball, sending it crashing to the floor, where it burst open, splattering its gummy contents everywhere"
Opium storage and the skilful boys working in the opium plants

When they could move no more, they sat ... (pp. 66)
"When they could move no more, they sat on the edges of the tanks, stirring the dark ooze only with their feet. These seated men had more the look of ghouls than any living thing she had ever seen: their eyes glowed red in the dark and they appeared completely naked, their loincloths - if indeed they had any - being so steeped in the drug as to be indistinguishable from their skin. Almost as frightening were the white overseers who were patrolling the walkways - for not only were they coatless and hatless, with their sleeves rolled, but they were also armed with fearsome instruments: met-al scoops, glass ladles and long-handled rakes. When one of these overseers approached her she all but screamed; she heard him say something - what it was she did not wish to know, but the very shock of being spoken to by such a man sent her scurrying down the tunnel and out at the far end."
... and the white masters who made these systems move. They are still there.

cane-wielding overseers and his howls and shrieks went ... (pp. 67)
"cane-wielding overseers and his howls and shrieks went echoing through the vast, chilly chamber. The screams sent her hurrying after her relative and she caught up with him on the threshold of yet another of the factory’s chambers."

Deeti was not of a mind to ignore ... (pp. 67)
"Deeti was not of a mind to ignore these attacks. From the shelter of her sari, she snapped back: And who are you to speak to me like that? How would you earn your living if not for afeemkhors?"
How would you?

So finely honed was the system, with relays ... (pp. 67)
"So finely honed was the system, with relays of runners carrying precise measures of each ingredient to each seat, that the assemblers’ hands never had cause to falter: they lined the moulds in such a way as to leave half the moistened rotis hanging over the edge. Then, dropping in the balls of opium, they covered them with the overhanging wrappers, and coated them with poppy-trash before tapping them out again. It remained only for run-ners to arrive with the outer casing for each ball - two halves of an earthenware sphere. The ball being dropped inside, the halves were fitted into a neat little cannonshot, to hold safe this most lucrative of the British Empire’s products: thus would the drug travel the seas, until the casing was split open by a blow from a cleaver, in distant Maha-Chin. Dozens of the black containers passed through the assemblers’ hands every hour and were duly noted on a blackboard: Hukam Singh, who was not the most skilled among them, had once boasted to Deeti of having put together a hundred in a single day. But today Hukam Singh’s hands were no longer working and nor was he at his usual seat: Deeti spot-ted him as she entered the assembly room - he was lying on the floor with his eyes closed and he looked as if he had had some kind of seizure, for a thin line of bubbling spit was dribbling out of the corner of his mouth. Suddenly, Deeti was assailed by the sirdars who supervised the packaging room. What took you so long? … Don’t you know your husband is an afeemkhor? … Why do you send him here to work? … Do you want him to die?"
Whose health? Whose business? Whose profits? the same old story continues... There are computer coolies, there are call centre workers toiling their days out, and these take a toll on their health. Then there is the lure of money and the dream of making it. But at whose expenses?

On the way home, in Kalua’s cart, with ... (pp. 68)
"On the way home, in Kalua’s cart, with her husband’s head in her lap and her daughter’s fingers in her hand, she had eyes neither for Ghazipur’s forty-pillared palace nor for its me-morial to the departed Laat-Sahib. Her thoughts were now all for the future and how they would manage without her husband’s monthly pay. In thinking of this, the light dimmed in her eyes; even though nightfall was still a couple of hours away, she felt as if she were already enveloped in darkness."
Fear of the future in the native land. Why people think of migration

The altercation drew the attention of an English ... (pp. 68)
"The altercation drew the attention of an English agent, who waved the sirdars aside. Glancing from Hukam Singh’s prone body to Deeti, he said, quietly: Tumhara mard hai? Is he your husband? Although the Englishman’s Hindi was stilted, there was a kindly sound to his voice. Deeti nodded, lowering her head, and her eyes filled with tears as she listened to the sahib berating the sirdars: Hukam Singh was a sepoy in our army; he was a balamteer in Burma and was wounded fighting for the Company Bahadur. Do you think any of you are better than him? Shut your mouths and get back to work or I’ll whip you with my own chabuck. The cowed sirdars fell silent, stepping aside as four bearers stooped to lift Hukam Singh’s inert body off the floor. Deeti was following them out when the Englishman turned to say: Tell him he can have his job back whenever he wants"
The whip!

Looking across the river Jodu could count the ... (pp. 71)
"Looking across the river Jodu could count the flags of a dozen kingdoms and countries: Genoa, the Two Sicilies, France, Prussia, Holland, America, Venice. He had learnt to recog-nize them from Putli, who had pointed them out to him as they sailed past the Gardens; even though she herself had never left Bengal, she knew stories about the places from which they came. These tales had played no small part in nurturing his desire to see the roses of Basra and the port of Chin-kalan, where the great Faghfoor of Maha-chin held sway"
The dream of the other lands ...

Back in the sheeshmahal, a bottle of champagne ... (pp. 74)
"Back in the sheeshmahal, a bottle of champagne was waiting in a balty of muddy river water. Mr Doughty fell upon the wine with an expression of delight: ‘Simkin! Shahbash -just the thing.’ Pouring himself a glass, he gave Neel a broad wink: ‘My father used to say, “Hold a bottle by the neck and a woman by the waist. Never the other way around.” I’ll wager that would have rung a ganta or two with your own father, eh Roger Nil-Rotten -now he was quite the rascal, wasn’t he, your father?’ Neel gave him a chilly smile: repelled as he was by the pilot’s manner, he could not help reflecting on what a mercy it was that his ancestors had excluded wine and liquor from the list of things that could not be shared with unclean foreigners - it would be all but impos-sible, surely, to deal with them, if not for their drink? He would have liked another glass of simkin but he noticed, from the corner of his eye, that Parimal was making signals to indicate that dinner was ready. He took the folds of his dhoti into his hands. ‘Gentlemen, I am being given to believe that our repast has been readied.’ As he rose to his feet, the sheeshmahal’s velvet curtain was swept back to reveal a large, polished table, set in the English fashion, with knives, forks, plates and wineglasses. Two immense candelabra stood at either end, illuminating the settings; in the centre was an arrangement of wilted water lilies, piled together in such profusion that almost nothing could be seen of the vase that held them. There was no food on the table, for meals in the Raskhali household were served in the Bengali fashion, in successive courses."
The opulence with which the rich native treats the sahib!

I do not think it sits well on ... (pp. 80)
"I do not think it sits well on a Raja of Raskhali to moralize on the subject ‘And why not?’ said Neel, steeling himself for the affront that was sure to follow. ‘Pray explain, Mr Burnham.’ ‘Why not?’ Mr Burnham’s eyebrows rose. ‘Well, for the very good reason that everything you possess is paid for by opium - this budgerow, your houses, this food. Do you think you could afford any of this on the revenues of your estate and your half-starved coolie farmers? No, sir: it’s opium that’s given you all of this.’"
the moment of truth; and likewise, we have in our times, our own opium ...

As his eyes became accustomed to the dim ... (pp. 96)
"As his eyes became accustomed to the dim light, Jodu stepped warily into one of these pens and immediately stubbed his toe upon a heavy iron chain. Falling to his knees, he discovered that there were several such chains in the pen, nailed into the far beam: they ended in bracelet-like clasps, each fitted with eye-holes, for locks. The weight and heft of the chains made Jodu wonder what sort of cargo they were intended to restrain: it occurred to him that they might be meant for livestock -and yet the stench that permeated the hold was not that of cows, horses or goats; it was more a human odour, compounded of sweat, urine, excrement and vomit; the smell had leached so deep into the timbers as to have become ineradicable. He picked up one of the chains, and on looking more closely at the bracelet-like clasps, he became convinced that it was in-deed meant for a human wrist or ankle."
Human beings treated as inhuman animals enslaved in ships - an accidental discovery

Suddenly he remembered stories, told on the river, ... (pp. 96)
"Suddenly he remembered stories, told on the river, of devil-ships that would descend on the coast to kidnap entire villages - the victims were eaten alive, or so the rumour went. Like an invasion of ghosts, unnamed apprehensions rushed into his mind; he pushed himself into a corner and sat shivering, falling gradually into a trance-like state of shock"

As for Jodu, his eyes went from Paulette’s ... (pp. 101)
"As for Jodu, his eyes went from Paulette’s face to Zachary’s and he knew at once, from the stiffness of their attitudes, that something of significance had passed between them. Having lost everything he owned, he had no qualms in using their new-found friendship to his advantage. O ke bol to re, he said in Bengali to Paulette: Tell him to find me a place on this ship’s lashkar. Tell him I have nowhere to go, nowhere to live - and it’s their fault, for running down my boat … Here Zachary broke in. ‘What’s he saying?’ ‘He says that he would like to gain a place on this ship,’ said Paulette. ‘Now that his boat is destroyed, he has nowhere to go …’"

By this time Deeti had abandoned the thought ... (pp. 104)
"By this time Deeti had abandoned the thought of paying for a new roof with the pro-ceeds of her poppies: she would have been content to earn enough to provide provisions for the season, with perhaps a handful or two of cowries for other expenses. The best she could hope for, she knew, was to come away from the factory with a couple of silver rupees; with luck, depending on the prices in the bazar, she might then have two or three copper dumrees left - maybe even as much as an adhela, to spend on a new sari for Kabutri. But a rude surprise was waiting at the Carcanna: after her gharas of opium had been weighed, counted and tested, Deeti was shown the account book for Hukam Singh’s plot of land. It turned out that at the start of the season, her husband had taken a much larger advance than she had thought: now, the meagre proceeds were barely enough to cover his debt. She looked disbelievingly at the discoloured coins that were laid before her: Aho se ka karwat? she cried. Just six dams for the whole harvest? It’s not enough to feed a child, let alone a family. The muharir behind the counter was a Bengali, with heavy jowls and a cataract of a frown. He answered her not in her native Bhojpuri, but in a mincing, citified Hindi: Do what others are doing, he snapped. Go to the moneylender. Sell your sons. Send them off to Mareech. It’s not as if you don’t have any choices. I have no sons to sell, said Deeti. Then sell your land, said the clerk, growing peevish. You people always come here and talk about being hungry, but tell me, who’s ever seen a peasant starve? You just like to com-plain, all the time khichir-michir …"
How natives themselves were their own enemies. Moving on to the metaphor for modern days, we have these people all over India in the modern times, acting at the behest of their masters without thinking twice wha tthey are doing!

Deeti resisted the offer till she thought of ... (pp. 104)
"Deeti resisted the offer till she thought of Kabutri: after all, the girl had just a few years left at home - why make her live through them in hunger? She gave in and agreed to place the impression of her thumb on the seth’s account book in exchange for six months’ worth of wheat, oil and gurh. Only as she was leaving did it occur to her to ask how much she owed and what the interest was. The seth’s answers took her breath away: his rates were such that her debt would double every six months; in a few years, all the land would be forfeit. Better to eat weeds than to take such a loan: she tried to return the goods but it was too late. I have your thumbprint now, said the seth, gloating. There’s nothing to be done"

Once the idea had been planted in her ... (pp. 106)
"Once the idea had been planted in her mind, Deeti could think of little else: better by far to die a celebrated death than to be dependent on Chandan Singh, or even to return to her own village, to live out her days as a shameful burden on her brother and her kin. The more she thought about it, the more persuasive the case - even where it concerned Kabutri. It was not as if she could promise her daughter a better life by staying alive as the mistress and ‘keep’ of a man of no account, like Chandan Singh. Precisely because he was her daugh-ter’s natural father, he would never allow the girl to be the equal of his other children - and his wife would do everything in her power to punish the child for her parentage. If she re-mained here, Kabutri would be little more than a servant and working-woman for her cous-ins; far better to send her back to her brother’s village, to be brought up with his children -a lone child would not be a burden. Deeti had always got on well with her brother’s wife, and knew that she would treat her daughter well. When looked at in this way, it seemed to Deeti that to go on living would be nothing more than selfishness - she could only be an impediment to her daughter’s happiness"
plight of Deeti on many fronts, suicidal ideation to become a Sati, and thoughts of the girl child that she could grow up with her cousins in the village rather than languish. In case Deeti had to embrace the life of becoming a mistress with Chandan Singh

she learnt that some distant relatives were travelling ... (pp. 106)
"she learnt that some distant relatives were travelling to the village where she was born: they agreed read-ily when she asked them to deliver her daughter to the house of her brother, Havildar Kesri Singh, the sepoy. The boat was to leave in a few hours and the pressure of time made it possible for Deeti to remain dry-eyed and composed as she tied Kabutri’s scant few pieces of clothing in a bundle. Among her few remaining pieces of jewellery were an anklet and a bangle: these she fastened on her daughter, with instructions to hand them over to her aunt: She’ll look after them for you. Kabutri was overjoyed at the prospect of visiting her cousins and living in a household filled with children. How long will I stay there? she asked. Until your father gets better. I’ll come to get you. When the boat sailed away, with Kabutri in it, it was as if Deeti’s last connection with life had been severed. From that moment she knew no further hesitation: with her habitu-al care, she set about making plans for her own end."

Baboo Nob Kissin had not neglected to pur-sue ... (pp. 110)
"Baboo Nob Kissin had not neglected to pur-sue a few opportunities of his own. Since much of his work consisted in acting as an inter-mediary and facilitator, he had acquired, over time, a wide circle of friends and acquaint-ances, many of whom relied on him for advice in matters pecuniary and personal. In time, his role as adviser turned into a thriving moneylending operation, often resorted to by gen-tlefolk who were in need of a discreet and reliable source of funding. There were some who came to him also for help in matters still more intimate: abstinent in all things but food, Baboo Nob Kissin regarded the carnal appetites of others with the detached curiosity with which an astrologer might observe the movements of the stars. He was unfailingly attentive to the women who appealed to him for assistance - and they in turn found him easy to trust, knowing that his devotion to Taramony would prevent him from exacting favours for him-self. It was thus that Elokeshi had come to regard him as an indulgent and kindly uncle"
Nab kissin was a middleman

All at once, everything was clear and he ... (pp. 112)
"All at once, everything was clear and he knew why things had happened as they had: it was because the Ibis was to take him to the place where his temple would be built"
But Naba Kissan would join the ship because of his religious obsession or his obsession with Taramony. Another reason for migration.

This is a forged signature, sir. And there ... (pp. 115)
"This is a forged signature, sir. And there is a great deal of money at stake.’ ‘To write a man’s name is not the same, surely, as forging his signature?’ ‘That depends on the intent, sir, which is for the court to decide,’ said the Major.’You may be sure that you will be given ample opportunity to make your case.’ ‘And in the meanwhile?’ ‘You must permit me to accompany you to Lalbazar.’ ‘To the jail?’ said Neel. ‘Like a common criminal?’ ‘Hardly that,’ the Major said. ‘We will make sure of your comfort; in consideration of your place in native society, we will even allow you to receive food from home.’ Now, at last, it began to sink in that the inconceivable was about to happen: the Raja of Raskhali was to be taken away by the police and locked in prison. Certain as he was that he would be acquitted, Neel knew that his family’s reputation would never again be what it was, not after a crowd of neighbours had witnessed his arrest and forcible removal - all his relatives, his dependants, his son, even Elokeshi, would be mired in the shame. ‘Do we have to go now?’ Neel demurred. ‘Today? In front of all my people?’"
The deception and dishonour of the royalty

The carriage arrived at the end of the ... (pp. 117)
"The carriage arrived at the end of the lane, and as it was turning the corner, Neel swiv-elled in his seat to take a last look at his house. He could see only the roof of the Raskhali Rajbari, and on it, outlined against the dimming sky, his son’s head, leaning on a parapet, as if in wait: he recalled that he had said he would be back in ten minutes, and this seemed to him now the most unpardonable of all the lies in his life"
the rajah's journey to jail

She called to him, Kalua, come, don’t leave ... (pp. 120)
"She called to him, Kalua, come, don’t leave me alone in this unknown place, come here. But when he lay down, she too was afraid: all of a sudden she was aware of how cold her body was, after its long immersion, and of the sopping wetness of her white sari. She began to shiver, and her hand, shaking, came upon his and she knew that he too was trem-bling, and slowly their bodies inched closer: as each sought the other’s warmth, their damp, sodden clothing came unspooled, his langot and her sari. Now it was as though she was on the water again: she remembered his touch and how he had held her to his chest with his arm. On the side of her face that was pressed to his, she could feel the gentle abrasion of his unshaved cheek - on the other side, which was flattened against the deck, she could hear the whispering of the earth and the river, and they were saying to her that she was alive, alive, and suddenly it was as if her body was awake to the world as it had never been be-fore, flowing like the river’s waves, and as open and fecund as the reed-covered bank"
Deeti was rescued from funeral pyre by Kalua

the serang stood apart also because of his ... (pp. 125)
"the serang stood apart also because of his origins, which were obscure even to those who had served with him longest. But this again was not un-usual, for many of the lascars were itinerants and vagrants, who did not care to speak too much about their past; some didn’t even know where their origins lay, having been sold off as children to the ghat-serangs who supplied lascars to ocean-going vessels. These river-side crimps cared nothing about who their recruits were and where they came from; all hands were the same to them, and their gangs would kidnap naked urchins from the streets and bearded sadhus from ashrams; they would pay brothel-keepers to drug their clients and thugs to lie in wait for unwary pilgrims"
the lascars, the riff raffs, the serang and jodu who was picked up becaues Zachary realised that the ship destroyed Jodu's raft and he was compassionate. Also, Jodu and Paulette knew each other and that may have played a role too in Jodu's landing up in the ship

Baboo Nob Kissin realized that he would have ... (pp. 130)
"Baboo Nob Kissin realized that he would have to endure it for a while yet. His best hope of finding a place on the Ibis was to be sent out as the ship’s supercargo, and the job was unlikely to come his way, he knew, if he gave the appearance of having lost interest in his work. And this too he knew, that if Mr Burnham were to have the least suspicion that there was some heathenish intent behind his seeking of the post of supercargo, then that would put an abrupt end to the matter. So for the time being, Baboo Nob Kissin decided, it was imperative that he apply himself to his duties and display as few signs as possible of the momentous transformations that were taking place within him. This was no easy task, for no matter how closely he tried to keep to his accustomed routines, he was ever more conscious that everything had changed and that he was seeing the world in new, unexpected ways"

After the food had been devoured, they sat ... (pp. 135)
"After the food had been devoured, they sat a while under the shade of a tree, and Kalua gave her a detailed account of all that had happened. They had arrived on the far side of the river to find eight men waiting, along with one of the duffadar’s sub-agents. Right there, on the shore, the men had entered their names on paper girmits; after these agree-ments were sealed, they had each been given a blanket, several articles of clothing, and a round-bottomed brass lota. Then, to celebrate their new-found status as girmitiyas, they had been served a meal - it was the remains of this feast that had been handed to Kalua by the duffadar. The gift was not given without protest: none of the recruits were strangers to hun-ger, and replete though they might be, they had been shocked to see so much food being given away. But the duffadar had told them they needn’t worry; they would have their fill at every meal; from now on, until they arrived in Mareech, that was all they needed to do -to eat and grow strong"
How deeti and Kalua became girmitiyas (girmits == ?agreements)

‘Nowadays all are going for pilgrimage by ship. ... (pp. 143)
"‘Nowadays all are going for pilgrimage by ship. Pilgrims cannot lose caste - this can also be like that. Why not?’ ‘Well I don’t know,’ said Mr Burnham, with a sigh. ‘Frankly, I don’t have time to think about it right now, with this Raskhali case coming up.’ This was the time, Baboo Nob Kissin knew, to play his best card. ‘Regarding case, sir, can I kindly be permitted to forward one suggestion?’ ‘Why, certainly,’ said Mr Burnham.’As I recall, it was all your idea in the first place, ‘Yes, sir,’ said the gomusta with a nod, ‘it was myself only who suggested you this Baboo Nob Kissin took no little pride in having been the first to alert his employer to the advantages of acquiring the Raskhali estate: for some years, it had been rumoured that the East India Company was to relinquish its control on opium production in eastern India. Were that to happen, poppies might well become a plantation crop, like indigo or sugar-cane: with the demand rising annually in China, merchants who controlled their own pro-duction, rather than depending on small farmers, would stand to multiply their already as-tronomical profits. Although there was, as yet, no clear sign that the Company was ready to make the necessary concessions, a few far-sighted merchants had already started looking for sizeable chunks of land. When Mr Burnham began to make inquiries, it was Baboo Nob Kissin who reminded him that he need look no further than the hugely indebted Raskhali estate, which was already within his grasp. He was well acquainted with several crannies and mootsuddies in the Raskhali daftar, and they had kept him closely informed of all the young zemindar’s missteps: like them, he regarded the new Raja as a dilettante, who had his nose in the air and his head in the clouds, and he fully shared their opinion that any-one so foolish as to sign everything that was put before him, deserved to lose his fortune. Besides, the Rajas of Raskhali were well known to be bigoted, ritual-bound Hindus, who were dismissive of heterodox Vaishnavites like himself: people like that needed to be taught a lesson from time to time."
The reason for the gomusta such as Naba Kissen to go andboard te ship

It is in keeping with these precedents that ... (pp. 159)
"It is in keeping with these precedents that this court pronounces its sentence, which is that all your properties are to be seized and sold, to make good your debts, and that you yourself are to be transported to the penal settlement on the Mauritius Islands for a period of no less than seven years. So let it be recorded on this, the twentieth day of July, in the year of Our Lord, 1838 …"
Circumstance of Neel Rattan's exile to Mauritius

It was not because of Ah Fatt’s fluency ... (pp. 246)
"It was not because of Ah Fatt’s fluency that Neel’s vision of Canton became so vivid as to make it real: in fact, the opposite was true, for the genius of Ah Fatt’s descriptions lay in their elisions, so that to listen to him was a venture of collaboration, in which the things that were spoken of came gradually to be transformed into artefacts of a shared ima-gining. So did Neel come to accept that Canton was to his own city as Calcutta was to the villages around it - a place of fearful splendour and unbearable squalor, as generous with its pleasures as it was unforgiving in the imposition of hardship. In listening and prompt-ing, Neel began to feel that he could almost see with Ah Fatt’s eyes"
The lure of imagined lands

Despite the heat of the sun the Captain’s ... (pp. 265)
"Despite the heat of the sun the Captain’s words had chilled Paulette to the marrow. As she looked around her now, she could see that many of the girmitiyas were in a trance of fear: it was as if they had just woken to the realization that they were not only leaving home and braving the Black Water - they were entering a state of existence in which their waking hours would be ruled by the noose and the whip. She could see their eyes straying to the island nearby; it was so close that its attraction was almost irresistible. When a grizzled, middle-aged man began to babble, she knew by instinct that he was losing his struggle against the pull of land. Although forewarned, she was still among the first to scream when this man made a sudden turn, shoved a lascar aside, and vaulted over the deck rail"
The source of plight! Find a metaphor