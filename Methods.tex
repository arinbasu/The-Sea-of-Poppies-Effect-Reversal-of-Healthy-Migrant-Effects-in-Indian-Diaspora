\subsection*{Data Analysis}

Data for this paper came from three sources:

\begin{enumerate}
\item The Novel, "Sea of Poppies" by Amitav Ghosh
\item Data from the Survey of Indians and Asian Immigrants in the Auckland CDHB Region, New Zealand
\item Primary data from a cross sectional survey in Bangalore, India
\end{enumerate}

The three pieces of information were linked together to generate the hypothesis that life and migration related stresses might be responsible for the pattern of illnesses and health states observed among Indian migrants.

\subsection*{Steps of the content analysis of the Ibis Trilogy}

The authors conducted a content analysis and narrative textual review of the book, The Sea of Poppies. These themes were related to the lived experiences of the main characters of the novel to indicate the state of migration and the themes that predicated the migrant labour from India to  Mauritius during the colonial period. This was done to portray a vignette of migration and movement of Indian diaspora across the time periods to other parts of the world and the situations that might have compelled them to leave their homeland. These themes were then examined in the light of the second content analyses conducted from the narratives of the survey that was conducted. The steps of the survey are narrated in the following section.

\subsection*{Review of the Literature on data on Indian diaspora in New Zealand}

Data from the health survey conducted in the Auckland Region of New Zealand was used to identify the patterns of disease conditions and health states reported by the members of he Indian diaspora and data from the documents were closely analysed to abstract patterns of illnesses expressed. 

\subsection*{Survey Data Analysis}
A survey conducted in India by Dr Shrimati Das (SD, an author of this paper). The results from the two surveys will be combined to examine and test the hypothesis that in the process of migration both from one country to another and from one setting to another, changes occur that are related to stresses and possibly life situations and broken promises that eventually result in the health issues that are observed in the affected individuals.

The third approach for this paper will include a close reading of the novel, "Sea of Poppies" and "River of Smoke" and data will be extracted from these two novels to a text analysis software and patterns will be searched based on the previously agreed themes: deception, stressors, and xxxx (we need to specify the themes at this stage).

A concept map will then be constructed to develop the hypothesis a comparable pattern of migration occurs across cultural divides and can contribute to life stresses and eventually lead to the maladjustment that have been observed in the Indian diaspora community that have settled overseas who demonstrate a reversal or deviation from the trend of healthy migrant effect. 

\subsection{The Process of the Survey}
\textbf{Srimati to fill this}

\subsection{Abstraction of Data from the North Island Survey}



    