

\subsection{Results}

A BPO service center is traditionally defined as a 'physical location where calls are either placed or received in a high volume for sales, marketing, customer service, telemarketing, technical support for other specialized business activities.It is however best described as a place where hundreds or thousands of youngsters sitting together in a large room separated by cubical walls, always under stress for orders to be placed or complaints to be resolved. It operates to provide round the clock and year round service.

Global organizations have always preferred outsourcing Call Center Srvices in India, when compared to outsourcing to China, Phillipines, Malaysia and other Asian countries, because of a variety of advantages that other countries do not offer.India has been able to effectively meet the growing international demand for Call Center Outsourcing Services(CCOS)More and more global organizations have been outsourcing Call Centers to India because of India's time zone advantage.India's 12 hour time differences enable global organizations to provide their customers with 24x7x365 days services.

But these advantages have their negative long lasting health fall outs.76% Call Center employees in Bangalore work in day shift , much more than Hyderabad at 24% and Pune at 28%.Another factor for India's advantage in having the largest number of state of art Call Centers in the world is the low cost of hiring which is only one tenth of the salaries of developed world. Our suvey on this aspect brought out startling statistics regarding the feeling of 'salary inadequacy', perticularly in Bangalore,the ITES hub in Asia, where out of 150 quizzed, 102 respondents were dissatisfied.This may be because of the high cost of living in Bangalore, Pune, a relatively inexpensive city to live in, gave us a percentage of 66%, who are happy with the salary they draw as a BPO worker.This growing sense of dissatisfaction among the ITES/BPO is the norm of the day. A peculiar case of not being valued for their skills and attributes has seen rising health related problems like stress and migraine.An acute sense of being underpaid has given rise to psychological fallouts in behavioural patterns; 64% in Bangalore Call Centers feel they are underpaid, followed by 20% in Pune,48% in Hyderabad. 

A special segment which was surveyed here as an integral part of this project was Women's participation in Transnational Call Center operations informing larger issues such as Globalization,Economic development, Gender equity and cultural acclimatization. Tulon (2004, p730) contended that 'cultural geography remains wholly almost wholly daylight geography'. But the night scape which serves as the basis of many 'changed cultural landscape' today in the Call Centers needs to be looked into, thereby not only changing the definition of 'cultural geography' from 'daylight' to 'night scape', it also brings into sharp focus the society's mindset towards women working ' night out', which is the biggest 'cultural taboo' for Indian Women for staying 'outdoor' after sunset. Hence, this survey did observe an  acute sense of 'guilt' in working educated, skilled women workforce in ITES/BPO sector. This leads to a high tension filled behavioral outlook, contrary to the nature of Inian women.For example 68% of women in Bangalore (a highly cosmopolitan city with migrant population), 84% in Pune (a Highly literate state in India) and 32% in Hyderabad(a conservative state)are not comfortable working in the 'night scape' job profile, adversely affecting the status of their psychological well being.

A question posed to women workforce 'whether they have social acceptance' towards this changed 'night scape' existence, the survey brought out again some startling findings; 65% in Bangalore felt that they are 'guilty' of staying out, 75% experienced guilt in a conservative town like Hyderabad.  Duality between 'cultural existence' in the ITES/BPO workforce in India- a position as being one of either the 'specular' border workforce' or the 'syncretic border workforce- a 'grey zone existence' According to Skeins model of organizational culture, leading to 'acute functional segregation' (Cheggappa & Goyal, 2002)

Another fall out on 'health status' of women workforce is to develop a sense of 'inadequacy' in terms of 'quality time' spent with family.This is more in a comparatively conservative city like Pune. 40% in Bangalore, 72% in Pune and 32% in Hyderabad expressed that they fail to give sufficient time to spouse/children & friends.Again when we surveyed on the question of how much the/day/night shift affects their normal family life, it was surprising that Pune, a traditional city in India, tops the chart where 88% think that their life is too job cenetred and they are not able to give that much time as much they would like to. They had a great sense of 'letting down' their family members in advancing their career at the expense of their family life.52% in Bangalore, 88% in Pune, 56% in Hyderabad think their life is 'too job centered'. Moreover, the belief that they are not being able to be good wife, dutiful daughter-in-law, loving mother- all age old traditional stereotypical roles enacted by Indian women down the ages. Now suddenly, this demand on their on time day/night shift is literally creating havoc in their family life and this has direct impact on their wellbeing. Cases of 'work- centered' Hypertension, Hyperglycemia, Depression,and Nervous breakdown are on the rise in ITES/BPO Women workforce in India; 52% in Bangalore, 64% in Pune, 48% in Hyderabad think that their job is eating into their family life space. 

CALL CENTER BLUES:
Stress is usually sensed as the body readjusts to too much of pressure.scientists use the term 'Homeostasis'to define the psychological limits in which the body functions efficiently and comfortably.Stress disturbs homeostasis by creating a state of imbalance.Long hours of work, permanent night shifts, and incredible high work targets are inherent factors of Call Center jobs. These are other social factors emanating from night shifts cause a multi level stress among the Call Center employees. Over a long period of time the stress response begins to take a toll on the body. The puffy eyes, red and blotchy skin, and edgy temperament demonstrates the stressful state of this workforce.Weight loss, deterioration of eyesight, shin problems, insomnia are normally experienced by BPO employees.

One of the prime targets affected in this stressful condition is the Thymus gland which plays a key role in the body's immune system.A weakened immune system makes the workforce vulnerable to infection and that is why people under stress often experiencing bouts of cold and cough and attack of flue.A few BPO offices visited have their in-house doctors, but they cannot prevent stress bound illnesses, which is rampant among the ITES workforce because of the very nature of their job.In the course of our survey, when the BPO workers were interviewed, it was found that the incessant calls would hardly provide them breathing space in between, and  many a times, they take their food while talking over the phone. Towards the end, they all feel quite exhausted and crave to look forward to the end of the shift.

The shift timings keep changing depending on from where the projects calls come from calling for the the need to adjust to different sets of timings which again add to the stress, since body clock takes time to adjust.Sometimes, the much needed relaxation does not happen even during the holidays, which are coinciding with that of client countries and this makes the employees out of sync with their families.Some workers in ITES sectors also suffer from 'loss of identity' which adds to the stress level significantly. After speaking in American accent and responding to the anglicized names- when Sujata becomes Susan, Madhu becomes Mary for the overseas clients, they feel a strange sense of identity loss.When they have to revert to their original names at home as Sujata and Madhu, they find it difficult in many cases. it is being reported that the outsourcing backlash is getting abusive and ugly. over the year, ordinary citizens have become sensitive to the job loss in their countries because of outsourcing, thanks to furor raised by the Western media. Here the employees on the phone are subjected to the angry outbursts of the clients bordering on being racist and sexist.

When the strains of stressful state are almost same for men and women,the impact is more in the case of women. on one hand they have to deal with obscene calls, and invitations, and on the other hand, they have to live with the doubts and suspicions of neighbors, relatives and friends, about the nature of their jobs. Now with the times changing the social acceptance of BPO jobs has considerably lessened the stress because of society's outlook.

While conducting this aspect of health impact survey, a unique 15 points'Health and Work Stress Scale'was formulated. The findings have been very disturbing.Apart from the acute ones mentioned above, the heavy surveillance, the monotony of sitting for 8 hours or more at a computer, the routine life, lack of authority are also very much present. All of them invariably show symptoms like fatigue, stiff neck, back/headaches, impaired vision, and numbness in fingers, fever, asthma, sore throats, nausea, dizziness, rashes, kidney stones, and ulcers. This also affects their 'Bio-rhythm'.The picture that emerges is very bleak. 

These health issues led to innumerable cases of attrition, which is high in urban based ITES firms than in the two-tier cities across India. The general consensus is that this high rate of attrition can be checked by implementing concrete measures in the area of 'Health Management'

World wide IT companies come to India to 'Body Shop' of highly skilled, educated workforce, but our youngsters are giving into this euphoria at what expense? Just read this fact: Entry level graduates in India are paid between INR 12,000 to INR 15,000 ($150-$200) a month, about a tenth of what their US counterparts earn. In the Phillipines, entry level BPO employees earn about $400- $650 a month, much more than an Indian. This is why 'Body Shoppers' come to India.



This project has taken steps to enumerate the experiences of 'New Age Outbound Skilled Labour Migrants' in the IT sectors from India to outside world, a new Indian diaspora  workforce in the emerging Buisness Process Outsourcing(BPO) and Information Technology Enabled Services(ITES) working in customer care services for globally outsourced entities. The four Indian States (Karnataka/Delhi/Maharashtra/Andhra Prodesh) case studies and survey giving certain insight into their 'health status' due to the changes in their life and working patterns apart from many other parameters.The findings through survey have been appalling.

Methodology:
This survey was conducted in 4 parts undertaken by the School of Management, Mount Carmel College, Bangalore on behalf of University Grants Commission, New Delhi as a part of a Major  Project Study by conducting on spot survey in reputed ITES/BPO organizations like Hewlett Packard(HP), Infosys,Dell in Bangalore(Karnataka).The Infosys/BPO system at Pune (Maharashtra), Wipro at Delhi, Infosys at Hyderabad (Andhra Prodesh).These states were represented by 150 respondents each all in the age group of 18-24 years. The questionnaires asked the respondents to rate their city/firm/job expectation/health parameters/ service providers/infrastructure/leisure encompassing 44 different parameters and give an overall rating on a scale of 1 to 5 where 1 represented very less and 5 very good.The block of 44 questions were broken up into 3 major areas of concern:
a. Language Competency
b. Gender Representation
c. Impact on Health
... through factor analysis a regression analysis was then done to see which factors weighed down heavily in determining ITES/BPO workforce overall rating.The weighted average scores of these three factors then yielded the overall average scores of these three factors. The Survey asked the same set of questions to 150 work-forces in each of the four cities.

An exploratory field work  from January 2010 to April 2012, involving interviews and observations, 25 formal semi-structured interviews with calling agent were conducted in English and each lasted about one hour each. Respondents were chosen randomly, samples were not gender specific but according to occupational level. Most of the respondents were young, highly educated and a mix of rural and urban.

Finally bringing the relevance of the study to the present, a few facts need to be mentioned. A case study conducted by Jayapal Dinesh Raja, Sanjeev kumar Bhasin for The Indian Journal of Community Medicine, on 'The Physical & Mental Health of Call Center Employees'- Vol, 39/Issue 3/page-175-177, 2014, in New Delhi, found that BPO employees were more health affected ,i.e more stressed 58.3% vs 19.3% ,more depressed 62.9% vs 4,6% and more anxious 33.9%vs 1.4% as compared to non BPO workers. But it is to the credit of the multinationals and humongous IT companies who very well are aware of the status of employees health and how they are being adversely affected due to this dramatic change in their life styles. As it is almost impossible to reduce the workload and increase the time, the only option left is to increase the energy level within the employees. For this almost all IT companies today have a 'Destress Zone', where one is compulsorily made to meditate and practise yoga. Indic spirituality offers plenty of ways of rejuvenating oneself. It's time the ITES/BPO workforce learn the art of dropping the world for a few minutes every day before it drops them. And its rejuvenating effect can only make the workforce better players in the world of information & Technology. Let us hope they learn how to overcome stress, killing stress before stress kills this entire IT enabled generation of highly skilled and highly educated workforce.

According to NASSCOM , in 2015, India's Information Technology & Buisness Process Management (IT-BPM)industry will add $12-15 billion incremental revenue, to existing industry revenues of & 118 billion .NASSCOM reports also stated that the industry added 160,000 employees , out of this women had a 32 % share and provide direct employment to 3.1 million people and indirect employment to 10 million people- a single job created directly in ITES/BPO segment results in additional three jobs indirectly in the related services such as security, transport & catering etc thereby creating more than 8.1 million jobs by the end of 2015 making India World's largest employable pool. India is the only country to offer full spectrum of ITES/BPO services, add to that India continuing to be the lowest cost location for IT domain.

                      BUT AT WHAT COST!!!
                     
 References: 
 
 *Health statistics quoted from on site survey available at University Grants Commission, Major Project submitted in 2013 on "Language Competency, Gender Representation, Cultural Acclimatization In the ITES/BPO Workforce - A Study of 5 States in India'
 *Statistics available at NASSCOM-McKinsey report- http//www.mcKinsey.com/location/India/mcKinseyon india/pdf/nasscom