\subsection{Results}

A BPO service center is traditionally defined as a 'physical location where calls are either placed or received in a high volume for sales, marketing, customer service, telemarketing, technical support for other specialized business activities.It is however best described as a place where hundreds or thousands of youngsters sitting together in a large room separated by cubical walls, always under stress for orders to be placed or complaints to be resolved. It operates to provide round the clock and year round service.

Global organizations have always preferred outsourcing Call Center Srvices in India, when compared to outsourcing to China, Phillipines, Malaysia and other Asian countries, because of a variety of advantages that other countries do not offer.India has been able to effectively meet the growing international demand for Call Center Outsourcing Services(CCOS)   

This project undertakes to enumerate the experiences of workforce in the emerging Buisness Process Outsourcing(BPO) and Information Technology Enabled Services(ITES) working in customer care services for globally outsourced entities. The four Indian States (Karnataka/Delhi/Maharashtra/Andhra Prodesh) case studies and survey giving certain insight into their 'health impact' due to the changes in their life and working patterns apart from many other parameters.

Methodology:
This survey was conducted in 4 parts undertaken by the School of Management, Mount Carmel College, Bangalore on behalf of University Grants Commission, New Delhi as a part of a Major  Project Study by conducting on spot survey in reputed ITES/BPO organizations like Hewlett Packard(HP), Infosys,Dell in Bangalore(Karnataka).The Infosys/BPO system at Pune (Maharashtra), Wipro at Delhi, Infosys at Hyderabad (Andhra Prodesh).These states were represented by 150 respondents each all in the age group of 18-24 years. The questionnaires asked the respondents to rate their city/firm/job expectation/health parameters/ service providers/infrastructure/leisure encompassing 44 different parameters and give an overall rating on a scale of 1 to 5 where 1 represented very less and 5 very good.The block of 44 questions were broken up into 3 major areas of concern:
a. Language Competency
b. Gender Representation
c. Impact on Health
... through factor analysis a regression analysis was then done to see which factors weighed down heavily in determining ITES/BPO workforce overall rating.The weighted average scores of these three factors then yielded the overall average scores of these three factors. The Survey asked the same set of questions to 150 work-forces in each of the four cities.

An exploratory field work  from January 2010 to April 2012, involving interviews and observations, 25 formal semi-structured interviews with calling agent were conducted in English and each lasted about one hour each. Respondents were chosen randomly, samples were not gender specific but according to occupational level. Most of the respondents were young, highly educated and a mix of rural and urban.