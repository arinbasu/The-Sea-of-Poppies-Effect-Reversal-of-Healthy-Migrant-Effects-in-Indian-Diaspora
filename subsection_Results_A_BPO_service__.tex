\subsection{Results}

A BPO service center is traditionally defined as a 'physical location where calls are either placed or received in a high volume for sales, marketing, customer service, telemarketing, technical support for other specialized business activities.It is however best described as a place where hundreds or thousands of youngsters sitting together in a large room separated by cubical walls, always under stress for orders to be placed or complaints to be resolved. It operates to provide round the clock and year round service.

Global organizations have always preferred outsourcing Call Center Srvices in India, when compared to outsourcing to China, Phillipines, Malaysia and other Asian countries, because of a variety of advantages that other countries do not offer.India has been able to effectively meet the growing international demand for Call Center Outsourcing Services(CCOS)More and more global organizations have been outsourcing Call Centers to India because of India's time zone advantage.India's 12 hour time differences enable global organizations to provide their customers with 24x7x365 days services.

But these advantages have their negative long lasting health fall outs.76% Call Center employees in Bangalore work in day shift , much more than Hyderabad at 24% and Pune at 28%.Another factor for India's advantage in having the largest number of state of art Call Centers in the world is the low cost of hiring which is only one tenth of the salaries of developed world. Our suvey on this aspect brought out startling statistics regarding the feeling of 'salary inadequacy', perticularly in Bangalore,the ITES hub in Asia, where out of 150 quizzed, 102 respondents were dissatisfied.This may be because of the high cost of living in Bangalore, Pune, a relatively inexpensive city to live in, gave us a percentage of 66%, who are happy with the salary they draw as a BPO worker.This growing sense of dissatisfaction among the ITES/BPO is the norm of the day. A peculiar case of not being valued for their skills and attributes has seen rising health related problems like stress and migraine.An acute sense of being underpaid has given rise to psychological fallouts in behavioural patterns; 64% in Bangalore Call Centers feel they are underpaid, followed by 20% in Pune,48% in Hyderabad. 

A special segment which was surveyed here as an integral part of this project was Women's participation in Transnational Call Center operations informing larger issues such as Globalization,Economic development, Gender equity and cultural acclimatization. Tulon (2004, p730) contended that 'cultural geography remains wholly almost wholly daylight geography'. But the night scape which serves as the basis of many 'changed cultural landscape' today in the Call Centers needs to be looked into, thereby not only changing the definition of 'cultural geography' from 'daylight' to 'night scape', it also brings into sharp focus the society's mindset towards women working ' night out', which is the biggest 'cultural taboo' for Indian Women for staying 'outdoor' after sunset. Hence, this survey did observe an  acute sense of 'guilt' in working educated, skilled women workforce in ITES/BPO sector. This leads to a high tension filled behavioral outlook, contrary to the nature of Inian women.For example 68% of women in Bangalore (a highly cosmopolitan city with migrant population), 84% in Pune (a Highly literate state in India) and 32% in Hyderabad(a conservative state)are not comfortable working in the 'night scape' job profile, adversly affecting the status of their psychological well being.

This project undertakes to enumerate the experiences of workforce in the emerging Buisness Process Outsourcing(BPO) and Information Technology Enabled Services(ITES) working in customer care services for globally outsourced entities. The four Indian States (Karnataka/Delhi/Maharashtra/Andhra Prodesh) case studies and survey giving certain insight into their 'health status' due to the changes in their life and working patterns apart from many other parameters.

Methodology:
This survey was conducted in 4 parts undertaken by the School of Management, Mount Carmel College, Bangalore on behalf of University Grants Commission, New Delhi as a part of a Major  Project Study by conducting on spot survey in reputed ITES/BPO organizations like Hewlett Packard(HP), Infosys,Dell in Bangalore(Karnataka).The Infosys/BPO system at Pune (Maharashtra), Wipro at Delhi, Infosys at Hyderabad (Andhra Prodesh).These states were represented by 150 respondents each all in the age group of 18-24 years. The questionnaires asked the respondents to rate their city/firm/job expectation/health parameters/ service providers/infrastructure/leisure encompassing 44 different parameters and give an overall rating on a scale of 1 to 5 where 1 represented very less and 5 very good.The block of 44 questions were broken up into 3 major areas of concern:
a. Language Competency
b. Gender Representation
c. Impact on Health
... through factor analysis a regression analysis was then done to see which factors weighed down heavily in determining ITES/BPO workforce overall rating.The weighted average scores of these three factors then yielded the overall average scores of these three factors. The Survey asked the same set of questions to 150 work-forces in each of the four cities.

An exploratory field work  from January 2010 to April 2012, involving interviews and observations, 25 formal semi-structured interviews with calling agent were conducted in English and each lasted about one hour each. Respondents were chosen randomly, samples were not gender specific but according to occupational level. Most of the respondents were young, highly educated and a mix of rural and urban.