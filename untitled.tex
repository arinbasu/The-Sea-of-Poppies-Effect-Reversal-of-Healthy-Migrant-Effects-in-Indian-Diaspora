\textit{Summary of the Paper} 

The purpose of this paper is to write about the excessive stress related diseases that are observed in the Indian diaspora settled in many different parts of the world. We shall also examine the health effects reported in a survey of Indian professionals who work in call centres throughout India as these indicate a form of internal migration. 

We hypothesise that a persistent pattern of unrealised promise, both in internal migration from villages or small towns to larger cities, and within larger cities from one set of circumstances to another as well as unrealised dreams and unfulfileed promises may underlie the atypical health effects that seem to reverse the "healthy migrant effect" among Indian diaspora that settle in different parts of the world, specifically reviewing evidence from New Zealand and evidence from a cross sectional survey in Bangalore. 

We hypothesise that there are similarities between the pattern of migration abroad in specific countries and job patterns for middle to lower-middle class non-technically educated Indians (excluding Engineeers and doctors, and other highly technical well placed jobs), and the transmigration of Indian labourers in the nineteenth centuries as described in the trilogy "Sea of Poppies". Drawing on the story of the migration detailed in the Sea of Poppies and current migration patterns, we lead to the hypothesis of the consistent pattern of reversal of healthy migrant effect seen in Indian settlers in developed countries. 
    